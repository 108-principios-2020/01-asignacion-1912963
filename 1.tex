% Options for packages loaded elsewhere
\PassOptionsToPackage{unicode}{hyperref}
\PassOptionsToPackage{hyphens}{url}
%
\documentclass[
]{article}
\usepackage{lmodern}
\usepackage{amssymb,amsmath}
\usepackage{ifxetex,ifluatex}
\ifnum 0\ifxetex 1\fi\ifluatex 1\fi=0 % if pdftex
  \usepackage[T1]{fontenc}
  \usepackage[utf8]{inputenc}
  \usepackage{textcomp} % provide euro and other symbols
\else % if luatex or xetex
  \usepackage{unicode-math}
  \defaultfontfeatures{Scale=MatchLowercase}
  \defaultfontfeatures[\rmfamily]{Ligatures=TeX,Scale=1}
\fi
% Use upquote if available, for straight quotes in verbatim environments
\IfFileExists{upquote.sty}{\usepackage{upquote}}{}
\IfFileExists{microtype.sty}{% use microtype if available
  \usepackage[]{microtype}
  \UseMicrotypeSet[protrusion]{basicmath} % disable protrusion for tt fonts
}{}
\makeatletter
\@ifundefined{KOMAClassName}{% if non-KOMA class
  \IfFileExists{parskip.sty}{%
    \usepackage{parskip}
  }{% else
    \setlength{\parindent}{0pt}
    \setlength{\parskip}{6pt plus 2pt minus 1pt}}
}{% if KOMA class
  \KOMAoptions{parskip=half}}
\makeatother
\usepackage{xcolor}
\IfFileExists{xurl.sty}{\usepackage{xurl}}{} % add URL line breaks if available
\IfFileExists{bookmark.sty}{\usepackage{bookmark}}{\usepackage{hyperref}}
\hypersetup{
  pdftitle={1.R},
  pdfauthor={Usuario},
  hidelinks,
  pdfcreator={LaTeX via pandoc}}
\urlstyle{same} % disable monospaced font for URLs
\usepackage[margin=1in]{geometry}
\usepackage{color}
\usepackage{fancyvrb}
\newcommand{\VerbBar}{|}
\newcommand{\VERB}{\Verb[commandchars=\\\{\}]}
\DefineVerbatimEnvironment{Highlighting}{Verbatim}{commandchars=\\\{\}}
% Add ',fontsize=\small' for more characters per line
\usepackage{framed}
\definecolor{shadecolor}{RGB}{248,248,248}
\newenvironment{Shaded}{\begin{snugshade}}{\end{snugshade}}
\newcommand{\AlertTok}[1]{\textcolor[rgb]{0.94,0.16,0.16}{#1}}
\newcommand{\AnnotationTok}[1]{\textcolor[rgb]{0.56,0.35,0.01}{\textbf{\textit{#1}}}}
\newcommand{\AttributeTok}[1]{\textcolor[rgb]{0.77,0.63,0.00}{#1}}
\newcommand{\BaseNTok}[1]{\textcolor[rgb]{0.00,0.00,0.81}{#1}}
\newcommand{\BuiltInTok}[1]{#1}
\newcommand{\CharTok}[1]{\textcolor[rgb]{0.31,0.60,0.02}{#1}}
\newcommand{\CommentTok}[1]{\textcolor[rgb]{0.56,0.35,0.01}{\textit{#1}}}
\newcommand{\CommentVarTok}[1]{\textcolor[rgb]{0.56,0.35,0.01}{\textbf{\textit{#1}}}}
\newcommand{\ConstantTok}[1]{\textcolor[rgb]{0.00,0.00,0.00}{#1}}
\newcommand{\ControlFlowTok}[1]{\textcolor[rgb]{0.13,0.29,0.53}{\textbf{#1}}}
\newcommand{\DataTypeTok}[1]{\textcolor[rgb]{0.13,0.29,0.53}{#1}}
\newcommand{\DecValTok}[1]{\textcolor[rgb]{0.00,0.00,0.81}{#1}}
\newcommand{\DocumentationTok}[1]{\textcolor[rgb]{0.56,0.35,0.01}{\textbf{\textit{#1}}}}
\newcommand{\ErrorTok}[1]{\textcolor[rgb]{0.64,0.00,0.00}{\textbf{#1}}}
\newcommand{\ExtensionTok}[1]{#1}
\newcommand{\FloatTok}[1]{\textcolor[rgb]{0.00,0.00,0.81}{#1}}
\newcommand{\FunctionTok}[1]{\textcolor[rgb]{0.00,0.00,0.00}{#1}}
\newcommand{\ImportTok}[1]{#1}
\newcommand{\InformationTok}[1]{\textcolor[rgb]{0.56,0.35,0.01}{\textbf{\textit{#1}}}}
\newcommand{\KeywordTok}[1]{\textcolor[rgb]{0.13,0.29,0.53}{\textbf{#1}}}
\newcommand{\NormalTok}[1]{#1}
\newcommand{\OperatorTok}[1]{\textcolor[rgb]{0.81,0.36,0.00}{\textbf{#1}}}
\newcommand{\OtherTok}[1]{\textcolor[rgb]{0.56,0.35,0.01}{#1}}
\newcommand{\PreprocessorTok}[1]{\textcolor[rgb]{0.56,0.35,0.01}{\textit{#1}}}
\newcommand{\RegionMarkerTok}[1]{#1}
\newcommand{\SpecialCharTok}[1]{\textcolor[rgb]{0.00,0.00,0.00}{#1}}
\newcommand{\SpecialStringTok}[1]{\textcolor[rgb]{0.31,0.60,0.02}{#1}}
\newcommand{\StringTok}[1]{\textcolor[rgb]{0.31,0.60,0.02}{#1}}
\newcommand{\VariableTok}[1]{\textcolor[rgb]{0.00,0.00,0.00}{#1}}
\newcommand{\VerbatimStringTok}[1]{\textcolor[rgb]{0.31,0.60,0.02}{#1}}
\newcommand{\WarningTok}[1]{\textcolor[rgb]{0.56,0.35,0.01}{\textbf{\textit{#1}}}}
\usepackage{graphicx,grffile}
\makeatletter
\def\maxwidth{\ifdim\Gin@nat@width>\linewidth\linewidth\else\Gin@nat@width\fi}
\def\maxheight{\ifdim\Gin@nat@height>\textheight\textheight\else\Gin@nat@height\fi}
\makeatother
% Scale images if necessary, so that they will not overflow the page
% margins by default, and it is still possible to overwrite the defaults
% using explicit options in \includegraphics[width, height, ...]{}
\setkeys{Gin}{width=\maxwidth,height=\maxheight,keepaspectratio}
% Set default figure placement to htbp
\makeatletter
\def\fps@figure{htbp}
\makeatother
\setlength{\emergencystretch}{3em} % prevent overfull lines
\providecommand{\tightlist}{%
  \setlength{\itemsep}{0pt}\setlength{\parskip}{0pt}}
\setcounter{secnumdepth}{-\maxdimen} % remove section numbering

\title{1.R}
\author{Usuario}
\date{2020-02-05}

\begin{document}
\maketitle

\begin{Shaded}
\begin{Highlighting}[]
\KeywordTok{library}\NormalTok{(repmis)}
\NormalTok{conjunto <-}\StringTok{ }\KeywordTok{source_data}\NormalTok{(}\StringTok{"https://www.dropbox.com/s/hmsf07bbayxv6m3/cuadro1.csv?dl=1"}\NormalTok{)}
\end{Highlighting}
\end{Shaded}

\begin{verbatim}
## Downloading data from: https://www.dropbox.com/s/hmsf07bbayxv6m3/cuadro1.csv?dl=1
\end{verbatim}

\begin{verbatim}
## SHA-1 hash of the downloaded data file is:
## 2bdde4663f51aa4198b04a248715d0d93498e7ba
\end{verbatim}

\begin{Shaded}
\begin{Highlighting}[]
\CommentTok{# media -------------------------------------------------------------------}


\KeywordTok{mean}\NormalTok{(conjunto}\OperatorTok{$}\NormalTok{Altura)}
\end{Highlighting}
\end{Shaded}

\begin{verbatim}
## [1] 13.9432
\end{verbatim}

\begin{Shaded}
\begin{Highlighting}[]
\KeywordTok{mean}\NormalTok{(conjunto}\OperatorTok{$}\NormalTok{Diametro)}
\end{Highlighting}
\end{Shaded}

\begin{verbatim}
## [1] 15.794
\end{verbatim}

\begin{Shaded}
\begin{Highlighting}[]
\KeywordTok{mean}\NormalTok{(conjunto}\OperatorTok{$}\NormalTok{Vecinos)}
\end{Highlighting}
\end{Shaded}

\begin{verbatim}
## [1] 3.34
\end{verbatim}

\begin{Shaded}
\begin{Highlighting}[]
\CommentTok{# altura  -----------------------------------------------------------------}


\NormalTok{H.media <-}\KeywordTok{subset}\NormalTok{(conjunto, conjunto}\OperatorTok{$}\NormalTok{Altura }\OperatorTok{<=}\StringTok{ }\FloatTok{13.9432}\NormalTok{)}

\NormalTok{H}\FloatTok{.16}\NormalTok{ <-}\StringTok{ }\KeywordTok{subset}\NormalTok{(conjunto,conjunto}\OperatorTok{$}\NormalTok{Altura }\OperatorTok{<}\StringTok{ }\FloatTok{16.5}\NormalTok{)}


\CommentTok{# Vecinos -----------------------------------------------------------------}

\NormalTok{vecinos_}\DecValTok{3}\NormalTok{ <-}\StringTok{ }\KeywordTok{subset}\NormalTok{(conjunto,conjunto}\OperatorTok{$}\NormalTok{Vecinos }\OperatorTok{<=}\StringTok{ }\DecValTok{3}\NormalTok{) }
\NormalTok{vecinos_}\DecValTok{4}\NormalTok{ <-}\StringTok{ }\KeywordTok{subset}\NormalTok{(conjunto,conjunto}\OperatorTok{$}\NormalTok{Vecinos }\OperatorTok{>}\StringTok{ }\DecValTok{4}\NormalTok{)}


\CommentTok{# Diámetro ----------------------------------------------------------------}

\NormalTok{DBH_media <-}\StringTok{ }\KeywordTok{subset}\NormalTok{(conjunto, conjunto}\OperatorTok{$}\NormalTok{Diametro }\OperatorTok{<}\StringTok{ }\FloatTok{15.794}\NormalTok{)}
\NormalTok{DBH_}\DecValTok{16}\NormalTok{ <-}\StringTok{ }\KeywordTok{subset}\NormalTok{(conjunto, conjunto}\OperatorTok{$}\NormalTok{Diametro }\OperatorTok{>}\StringTok{ }\DecValTok{16}\NormalTok{)}


\CommentTok{# Especie -----------------------------------------------------------------}

\NormalTok{Cedro_rojo <-}\StringTok{ }\KeywordTok{subset}\NormalTok{(conjunto, conjunto}\OperatorTok{$}\NormalTok{Especie }\OperatorTok{==}\StringTok{ "C"}\NormalTok{)}
\NormalTok{Diametrocedrorojo <-}\StringTok{ }\KeywordTok{subset}\NormalTok{(Cedro_rojo, Cedro_rojo}\OperatorTok{$}\NormalTok{Diametro }\OperatorTok{<=}\StringTok{ }\FloatTok{16.9}\NormalTok{)}
\NormalTok{Alturacedrorojo <-}\StringTok{ }\KeywordTok{subset}\NormalTok{(Cedro_rojo, Cedro_rojo}\OperatorTok{$}\NormalTok{Altura }\OperatorTok{>}\StringTok{ }\FloatTok{18.5}\NormalTok{)}

\NormalTok{Tsuga <-}\StringTok{ }\KeywordTok{subset}\NormalTok{(conjunto, conjunto}\OperatorTok{$}\NormalTok{Especie }\OperatorTok{==}\StringTok{ "H"}\NormalTok{)}
\NormalTok{Diametrotsuga <-}\StringTok{ }\KeywordTok{subset}\NormalTok{(Tsuga, Tsuga}\OperatorTok{$}\NormalTok{Diametro }\OperatorTok{<=}\StringTok{ }\FloatTok{16.9}\NormalTok{)}
\NormalTok{Alturatsuga <-}\StringTok{ }\KeywordTok{subset}\NormalTok{(Tsuga, Tsuga}\OperatorTok{$}\NormalTok{Altura }\OperatorTok{>}\StringTok{ }\FloatTok{18.5}\NormalTok{)}

\NormalTok{Douglasia <-}\StringTok{ }\KeywordTok{subset}\NormalTok{(conjunto, conjunto}\OperatorTok{$}\NormalTok{Especie }\OperatorTok{==}\StringTok{ "F"}\NormalTok{)}
\NormalTok{Diametrodou <-}\StringTok{ }\KeywordTok{subset}\NormalTok{(Douglasia, Douglasia}\OperatorTok{$}\NormalTok{Diametro }\OperatorTok{<=}\StringTok{ }\FloatTok{16.9}\NormalTok{)}
\NormalTok{Alturadou <-}\StringTok{ }\KeywordTok{subset}\NormalTok{(Douglasia, Douglasia}\OperatorTok{$}\NormalTok{Altura }\OperatorTok{>=}\StringTok{ }\FloatTok{18.5}\NormalTok{)}



\CommentTok{# Histogramas -------------------------------------------------------------}

\KeywordTok{hist}\NormalTok{(conjunto}\OperatorTok{$}\NormalTok{Altura, }\DataTypeTok{col=}\StringTok{"yellow"}\NormalTok{,}\DataTypeTok{xlab =}\StringTok{"Altura"}\NormalTok{, }\DataTypeTok{ylab =} \StringTok{"frecuencia"}\NormalTok{, }\DataTypeTok{main =}\StringTok{"Histograma de altura"}\NormalTok{)}
\end{Highlighting}
\end{Shaded}

\includegraphics{1_files/figure-latex/unnamed-chunk-1-1.pdf}

\begin{Shaded}
\begin{Highlighting}[]
\KeywordTok{hist}\NormalTok{(conjunto}\OperatorTok{$}\NormalTok{Diametro, }\DataTypeTok{col=}\StringTok{"red"}\NormalTok{,}\DataTypeTok{xlab =}\StringTok{"Diametro"}\NormalTok{, }\DataTypeTok{ylab =} \StringTok{"frecuencia"}\NormalTok{, }\DataTypeTok{main =}\StringTok{"Histograma de Diametro"}\NormalTok{)}
\end{Highlighting}
\end{Shaded}

\includegraphics{1_files/figure-latex/unnamed-chunk-1-2.pdf}

\begin{Shaded}
\begin{Highlighting}[]
\KeywordTok{hist}\NormalTok{(conjunto}\OperatorTok{$}\NormalTok{Vecinos, }\DataTypeTok{col=}\StringTok{"orange"}\NormalTok{,}\DataTypeTok{xlab =}\StringTok{"Vecinos"}\NormalTok{, }\DataTypeTok{ylab =} \StringTok{"frecuencia"}\NormalTok{, }\DataTypeTok{main =}\StringTok{"Histograma de Vecinos"}\NormalTok{)}
\end{Highlighting}
\end{Shaded}

\includegraphics{1_files/figure-latex/unnamed-chunk-1-3.pdf}

\begin{Shaded}
\begin{Highlighting}[]
\CommentTok{# desviaciones estandar ---------------------------------------------------}

\KeywordTok{sd}\NormalTok{(conjunto}\OperatorTok{$}\NormalTok{Altura)}
\end{Highlighting}
\end{Shaded}

\begin{verbatim}
## [1] 2.907177
\end{verbatim}

\begin{Shaded}
\begin{Highlighting}[]
\KeywordTok{sd}\NormalTok{(conjunto}\OperatorTok{$}\NormalTok{Diametro)}
\end{Highlighting}
\end{Shaded}

\begin{verbatim}
## [1] 3.227017
\end{verbatim}

\begin{Shaded}
\begin{Highlighting}[]
\KeywordTok{sd}\NormalTok{(conjunto}\OperatorTok{$}\NormalTok{Vecinos)}
\end{Highlighting}
\end{Shaded}

\begin{verbatim}
## [1] 1.598596
\end{verbatim}

\end{document}
